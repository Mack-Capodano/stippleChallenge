% Options for packages loaded elsewhere
% Options for packages loaded elsewhere
\PassOptionsToPackage{unicode}{hyperref}
\PassOptionsToPackage{hyphens}{url}
\PassOptionsToPackage{dvipsnames,svgnames,x11names}{xcolor}
%
\documentclass[
  letterpaper,
  DIV=11,
  numbers=noendperiod]{scrartcl}
\usepackage{xcolor}
\usepackage{amsmath,amssymb}
\setcounter{secnumdepth}{-\maxdimen} % remove section numbering
\usepackage{iftex}
\ifPDFTeX
  \usepackage[T1]{fontenc}
  \usepackage[utf8]{inputenc}
  \usepackage{textcomp} % provide euro and other symbols
\else % if luatex or xetex
  \usepackage{unicode-math} % this also loads fontspec
  \defaultfontfeatures{Scale=MatchLowercase}
  \defaultfontfeatures[\rmfamily]{Ligatures=TeX,Scale=1}
\fi
\usepackage{lmodern}
\ifPDFTeX\else
  % xetex/luatex font selection
\fi
% Use upquote if available, for straight quotes in verbatim environments
\IfFileExists{upquote.sty}{\usepackage{upquote}}{}
\IfFileExists{microtype.sty}{% use microtype if available
  \usepackage[]{microtype}
  \UseMicrotypeSet[protrusion]{basicmath} % disable protrusion for tt fonts
}{}
\makeatletter
\@ifundefined{KOMAClassName}{% if non-KOMA class
  \IfFileExists{parskip.sty}{%
    \usepackage{parskip}
  }{% else
    \setlength{\parindent}{0pt}
    \setlength{\parskip}{6pt plus 2pt minus 1pt}}
}{% if KOMA class
  \KOMAoptions{parskip=half}}
\makeatother
% Make \paragraph and \subparagraph free-standing
\makeatletter
\ifx\paragraph\undefined\else
  \let\oldparagraph\paragraph
  \renewcommand{\paragraph}{
    \@ifstar
      \xxxParagraphStar
      \xxxParagraphNoStar
  }
  \newcommand{\xxxParagraphStar}[1]{\oldparagraph*{#1}\mbox{}}
  \newcommand{\xxxParagraphNoStar}[1]{\oldparagraph{#1}\mbox{}}
\fi
\ifx\subparagraph\undefined\else
  \let\oldsubparagraph\subparagraph
  \renewcommand{\subparagraph}{
    \@ifstar
      \xxxSubParagraphStar
      \xxxSubParagraphNoStar
  }
  \newcommand{\xxxSubParagraphStar}[1]{\oldsubparagraph*{#1}\mbox{}}
  \newcommand{\xxxSubParagraphNoStar}[1]{\oldsubparagraph{#1}\mbox{}}
\fi
\makeatother


\usepackage{longtable,booktabs,array}
\usepackage{calc} % for calculating minipage widths
% Correct order of tables after \paragraph or \subparagraph
\usepackage{etoolbox}
\makeatletter
\patchcmd\longtable{\par}{\if@noskipsec\mbox{}\fi\par}{}{}
\makeatother
% Allow footnotes in longtable head/foot
\IfFileExists{footnotehyper.sty}{\usepackage{footnotehyper}}{\usepackage{footnote}}
\makesavenoteenv{longtable}
\usepackage{graphicx}
\makeatletter
\newsavebox\pandoc@box
\newcommand*\pandocbounded[1]{% scales image to fit in text height/width
  \sbox\pandoc@box{#1}%
  \Gscale@div\@tempa{\textheight}{\dimexpr\ht\pandoc@box+\dp\pandoc@box\relax}%
  \Gscale@div\@tempb{\linewidth}{\wd\pandoc@box}%
  \ifdim\@tempb\p@<\@tempa\p@\let\@tempa\@tempb\fi% select the smaller of both
  \ifdim\@tempa\p@<\p@\scalebox{\@tempa}{\usebox\pandoc@box}%
  \else\usebox{\pandoc@box}%
  \fi%
}
% Set default figure placement to htbp
\def\fps@figure{htbp}
\makeatother





\setlength{\emergencystretch}{3em} % prevent overfull lines

\providecommand{\tightlist}{%
  \setlength{\itemsep}{0pt}\setlength{\parskip}{0pt}}



 


\KOMAoption{captions}{tableheading}
\makeatletter
\@ifpackageloaded{caption}{}{\usepackage{caption}}
\AtBeginDocument{%
\ifdefined\contentsname
  \renewcommand*\contentsname{Table of contents}
\else
  \newcommand\contentsname{Table of contents}
\fi
\ifdefined\listfigurename
  \renewcommand*\listfigurename{List of Figures}
\else
  \newcommand\listfigurename{List of Figures}
\fi
\ifdefined\listtablename
  \renewcommand*\listtablename{List of Tables}
\else
  \newcommand\listtablename{List of Tables}
\fi
\ifdefined\figurename
  \renewcommand*\figurename{Figure}
\else
  \newcommand\figurename{Figure}
\fi
\ifdefined\tablename
  \renewcommand*\tablename{Table}
\else
  \newcommand\tablename{Table}
\fi
}
\@ifpackageloaded{float}{}{\usepackage{float}}
\floatstyle{ruled}
\@ifundefined{c@chapter}{\newfloat{codelisting}{h}{lop}}{\newfloat{codelisting}{h}{lop}[chapter]}
\floatname{codelisting}{Listing}
\newcommand*\listoflistings{\listof{codelisting}{List of Listings}}
\makeatother
\makeatletter
\makeatother
\makeatletter
\@ifpackageloaded{caption}{}{\usepackage{caption}}
\@ifpackageloaded{subcaption}{}{\usepackage{subcaption}}
\makeatother
\usepackage{bookmark}
\IfFileExists{xurl.sty}{\usepackage{xurl}}{} % add URL line breaks if available
\urlstyle{same}
\hypersetup{
  pdftitle={Selection Bias \& Missing Data Challenge - Part 1},
  colorlinks=true,
  linkcolor={blue},
  filecolor={Maroon},
  citecolor={Blue},
  urlcolor={Blue},
  pdfcreator={LaTeX via pandoc}}


\title{Selection Bias \& Missing Data Challenge - Part 1}
\usepackage{etoolbox}
\makeatletter
\providecommand{\subtitle}[1]{% add subtitle to \maketitle
  \apptocmd{\@title}{\par {\large #1 \par}}{}{}
}
\makeatother
\subtitle{Blue Noise Stippling: Creating Art from Data}
\author{}
\date{}
\begin{document}
\maketitle


\section{🎨 Selection Bias \& Missing Data Challenge - Part
1}\label{selection-bias-missing-data-challenge---part-1}

\subsection{The Problem: Can Algorithms Create
Art?}\label{the-problem-can-algorithms-create-art}

\textbf{Core Question:} How can we convert a photograph into an
aesthetically pleasing pattern of dots that preserves the visual
information of the original image?

\textbf{The Challenge:} Blue noise stippling is a technique that
converts images into patterns of dots (stipples) using algorithms that
balance visual accuracy with spatial distribution. This challenge asks
you to implement a modified ``void and cluster'' algorithm that combines
importance sampling with blue noise distribution properties to create
stippling patterns that are both visually accurate and spatially
well-distributed.

\textbf{Our Approach:} We'll use a modified void-and-cluster algorithm
that: 1. Creates an importance map identifying visually important
regions 2. Uses a toroidal (periodic) Gaussian kernel for repulsion to
ensure blue noise properties 3. Iteratively selects points with minimum
energy 4. Balances image content importance with blue noise spatial
distribution

\subsection{Introduction to Blue Noise
Stippling}\label{introduction-to-blue-noise-stippling}

Blue noise stippling is a technique for converting images into a pattern
of dots (stipples) that preserves the visual information of the original
image while creating an aesthetically pleasing, evenly distributed
pattern. This method follows the approach described by
\href{https://bartwronski.com/2022/08/31/progressive-image-stippling-and-greedy-blue-noise-importance-sampling/}{Bart
Wronski}.

The method uses a modified ``void and cluster'' algorithm that combines
importance sampling with blue noise distribution properties to create
stippling patterns that are both visually accurate and spatially
well-distributed. This version uses \textbf{smooth extreme
downweighting} that selectively downweights very dark and very light
tones while preserving mid-tones, creating a more balanced distribution
of stipples across the image.

\subsection{Loading the Original
Image}\label{loading-the-original-image}

First, let's load an image that we'll convert to a blue noise stippling
pattern. You can use any image you'd like, but we'll demonstrate with
the provided example.

\subsubsection{Python}

\begin{figure}

{\centering \pandocbounded{\includegraphics[keepaspectratio]{index_files/figure-pdf/load-image-output-1.pdf}}

}

\caption{Original image before stippling}

\end{figure}%

\begin{verbatim}
Image shape: (251, 201)
Image size: 251x201 pixels
\end{verbatim}

\subsection{Importance Mapping}\label{importance-mapping}

Before applying the stippling algorithm, we create an \textbf{importance
map} that identifies which regions of the image should receive more
stipples. The importance map is computed by:

\begin{itemize}
\tightlist
\item
  \textbf{Brightness inversion}: The image brightness is inverted so
  that dark areas receive higher importance and thus more dots, while
  light areas receive fewer dots
\item
  \textbf{Extreme tone downweighting}: Smooth Gaussian functions
  downweight tones below 0.2 (very dark) and above 0.8 (very light),
  creating a gradual transition that preserves mid-tones
\item
  \textbf{Mid-tone boost}: A smooth Gaussian function centered on
  mid-tones provides a gradual increase in importance for mid-tone
  regions, ensuring they receive appropriate stippling density
\item
  \textbf{Selective and effective}: This approach ensures that stipples
  are distributed appropriately (more dots in dark areas and mid-tones,
  fewer in extreme dark/light areas) while maintaining good spatial
  distribution
\end{itemize}

\subsubsection{Python}

\subsection{Blue Noise Stippling
Algorithm}\label{blue-noise-stippling-algorithm}

The stippling algorithm uses a modified void-and-cluster approach that:

\begin{enumerate}
\def\labelenumi{\arabic{enumi}.}
\tightlist
\item
  Creates an importance map that identifies visually important regions
\item
  Initializes an energy field based on the importance map (higher
  importance → lower energy)
\item
  Uses a toroidal (periodic) Gaussian kernel for repulsion to ensure
  blue noise properties
\item
  Iteratively selects points with minimum energy
\item
  Adds Gaussian ``splats'' around selected points to prevent clustering
\item
  Balances image content importance with blue noise spatial distribution
\end{enumerate}

\subsubsection{Python}

\subsection{Preparing the Working
Image}\label{preparing-the-working-image}

Before generating the stippling pattern, we prepare the image by
resizing if necessary and computing the importance map.

\subsubsection{Python}

\phantomsection\label{prep-image}
\begin{verbatim}
Final image shape: (251, 201) (should be 2D for grayscale)
Importance map computed
\end{verbatim}

\subsection{Generating the Stippled
Image}\label{generating-the-stippled-image}

Now let's apply the stippling algorithm to create the blue noise
stippling pattern.

\subsubsection{Python}

\phantomsection\label{generate-stipple}
\begin{verbatim}
Generating blue noise stippling pattern...
Generated 4036 stipple points
Stipple pattern shape: (251, 201)
\end{verbatim}

\subsection{Displaying the Results}\label{displaying-the-results}

Let's visualize the original image, the importance map, and the stippled
version side by side for comparison.

\subsubsection{Python}

\begin{figure}

{\centering \pandocbounded{\includegraphics[keepaspectratio]{index_files/figure-pdf/display-results-output-1.pdf}}

}

\caption{Comparison of original image, importance map, and blue noise
stippling}

\end{figure}%

\subsection{Experimental Variations}\label{experimental-variations}

Let's experiment with different parameters and approaches to see how
they affect the stippling results.

\subsubsection{Darker Colors Focus}

This experiment modifies the importance map to strongly prioritize
darker areas, creating stipples that emphasize shadow regions.

\paragraph{Python}

\begin{verbatim}
Image shape: (251, 201)
Generating stippling pattern (dark focus)...
Generated 4036 stipple points
\end{verbatim}

\begin{figure}

{\centering \pandocbounded{\includegraphics[keepaspectratio]{index_files/figure-pdf/darker-colors-experiment-output-2.pdf}}

}

\caption{Stippling experiment focusing on darker colors}

\end{figure}%

\textbf{Changes \& Why:} - \textbf{No Brightness Inversion}: Removed
1.0-img inversion (dark areas naturally get higher importance) -
\textbf{Tighter Parameters}: extreme\_sigma=0.05 (sharper
downweighting), mid\_tone\_sigma=0.15 (focused boost) - \textbf{Enhanced
Mid-tone Boost}: Centered at 0.2 brightness with 80\% weight for dark
mid-tones - \textbf{Reduced Extreme Downweighting}: 0.15 (vs 0.2) to
maintain some contrast - \textbf{Why}: Strongly emphasizes dark areas
(jersey) while preserving mid-tone details, creating shadow-focused
stippling

\paragraph{}

\subsubsection{Edge-Enhanced Player
Focus}\label{edge-enhanced-player-focus}

This experiment uses edge detection to emphasize the player's silhouette
and features, making him stand out from the background through contrast
enhancement.

\paragraph{Python}

\begin{verbatim}
Image shape: (251, 201)
Generating edge-enhanced stippling pattern...
Generated 5045 stipple points
\end{verbatim}

\begin{figure}

{\centering \pandocbounded{\includegraphics[keepaspectratio]{index_files/figure-pdf/edge-enhanced-player-experiment-output-2.pdf}}

}

\caption{Edge-enhanced stippling to make the player pop out}

\end{figure}%

\textbf{Changes \& Why:} - \textbf{Edge Detection}: Added Sobel operator
edge detection to find boundaries - \textbf{Local Contrast}: Measured
variance in 5x5 windows for high-contrast areas - \textbf{Combined
Boost}: Edge + contrast signals added to base importance (60\% weight
each) - \textbf{Higher Density}: 10\% points (vs 8\%) for more detail on
player features - \textbf{Balanced Sigma}: 0.8 for good clustering
without over-smoothing - \textbf{Why}: Emphasizes player silhouette
through contrast enhancement, making him pop from background

\paragraph{}

\subsubsection{Contrast-Based Player
Isolation}\label{contrast-based-player-isolation}

This experiment creates an importance map based on local contrast
patterns, boosting areas where bright regions (player's skin) meet dark
regions (jersey), creating a silhouette effect.

\paragraph{Python}

\begin{verbatim}
Image shape: (251, 201)
Generating contrast-based player stippling pattern...
Generated 4036 stipple points
\end{verbatim}

\begin{figure}

{\centering \pandocbounded{\includegraphics[keepaspectratio]{index_files/figure-pdf/contrast-player-experiment-output-2.pdf}}

}

\caption{Contrast-based stippling to isolate the player silhouette}

\end{figure}%

\textbf{Changes \& Why:} - \textbf{Local Range Analysis}: Used 7x7
sliding windows to measure brightness range (max-min) - \textbf{Medium
Brightness Focus}: Gaussian boost centered on 0.5 brightness for
skin-jersey boundaries - \textbf{Dark Area Boost}: Additional emphasis
on pure dark areas (jersey) with tight sigma (0.15) - \textbf{Background
Suppression}: Reduced importance for very bright areas (background) by
70\% - \textbf{High Content Bias}: 0.95 (vs 0.9) for better shape
preservation - \textbf{Why}: Creates figure-ground separation by
isolating high-contrast player regions from uniform background

\paragraph{}

\subsubsection{High Density Stippling
(12\%)}\label{high-density-stippling-12}

This experiment uses a higher percentage of stipple points (12\% instead
of 8\%) to create denser stippling.

\paragraph{Python}

\begin{verbatim}
Generating high-density stippling pattern...
Generated 6054 stipple points
\end{verbatim}

\begin{figure}

{\centering \pandocbounded{\includegraphics[keepaspectratio]{index_files/figure-pdf/high-density-experiment-output-2.pdf}}

}

\caption{High density stippling experiment (12\% points)}

\end{figure}%

\textbf{Changes \& Why:} - \textbf{Higher Density}: percentage=0.12 (vs
0.08) for 50\% more stipple points - \textbf{Same Parameters}: All other
settings identical to baseline (sigma=0.9, content\_bias=0.9) -
\textbf{Why}: Creates denser, more detailed stippling with finer
granularity, revealing more image features

\paragraph{}

\subsubsection{Low Density Stippling
(5\%)}\label{low-density-stippling-5}

This experiment uses a lower percentage of stipple points (5\% instead
of 8\%) to create sparser stippling.

\paragraph{Python}

\begin{verbatim}
Generating low-density stippling pattern...
Generated 2522 stipple points
\end{verbatim}

\begin{figure}

{\centering \pandocbounded{\includegraphics[keepaspectratio]{index_files/figure-pdf/low-density-experiment-output-2.pdf}}

}

\caption{Low density stippling experiment (5\% points)}

\end{figure}%

\textbf{Changes \& Why:} - \textbf{Lower Density}: percentage=0.05 (vs
0.08) for 37\% fewer stipple points - \textbf{Same Parameters}: All
other settings identical to baseline (sigma=0.9, content\_bias=0.9) -
\textbf{Why}: Creates minimal, artistic stippling that captures
essential shapes with fewer points, emphasizing bold contrasts

\paragraph{}

\subsubsection{High Sigma Stippling
(σ=1.2)}\label{high-sigma-stippling-ux3c31.2}

This experiment uses a higher sigma value (1.2 instead of 0.9) for the
blue noise generation, creating more clustered stippling patterns.

\paragraph{Python}

\begin{verbatim}
Generating high-sigma stippling pattern...
Generated 4036 stipple points
\end{verbatim}

\begin{figure}

{\centering \pandocbounded{\includegraphics[keepaspectratio]{index_files/figure-pdf/high-sigma-experiment-output-2.pdf}}

}

\caption{High sigma stippling experiment (σ=1.2)}

\end{figure}%

\textbf{Changes \& Why:} - \textbf{Higher Sigma}: sigma=1.2 (vs 0.9) for
blue noise generation - \textbf{Same Parameters}: All other settings
identical to baseline (percentage=0.08, content\_bias=0.9) -
\textbf{Why}: Creates more clustered stippling with larger spacing
variations, producing a more organic, less uniform distribution

\paragraph{}

\subsubsection{Low Sigma Stippling
(σ=0.5)}\label{low-sigma-stippling-ux3c30.5}

This experiment uses a lower sigma value (0.5 instead of 0.9) for the
blue noise generation, creating more dispersed stippling patterns.

\paragraph{Python}

\begin{verbatim}
Generating low-sigma stippling pattern...
Generated 4036 stipple points
\end{verbatim}

\begin{figure}

{\centering \pandocbounded{\includegraphics[keepaspectratio]{index_files/figure-pdf/low-sigma-experiment-output-2.pdf}}

}

\caption{Low sigma stippling experiment (σ=0.5)}

\end{figure}%

\textbf{Changes \& Why:} - \textbf{Lower Sigma}: sigma=0.5 (vs 0.9) for
blue noise generation - \textbf{Same Parameters}: All other settings
identical to baseline (percentage=0.08, content\_bias=0.9) -
\textbf{Why}: Creates more dispersed stippling with smaller spacing
variations, producing a more uniform, grid-like distribution

\subsection{How the parameter changes affect the final
result}\label{how-the-parameter-changes-affect-the-final-result}

-The different parameters affect the result in different ways. Some
emphasize the dark jersey areas, some emphasize the player silhouette,
some emphasize the background, some emphasize the mid-tones. There are
also different levels of clustering done depending on parameter
settings. All these affect how much or little Drake pops out from the
background and his jersey. The best results came from the original, the
high density, and the player edge focus settings. These settings created
the most balanced and visually pleasing results. \#\#\#\#

\subsection{Progressive Stippling
Animation}\label{progressive-stippling-animation}

This section creates a GIF showing how the stippled image looks as more
points are added sequentially. We'll use the already-computed stippling
points to generate frames at increments of 100 points.

\subsubsection{Python}

\phantomsection\label{progressive-stippling}
\begin{verbatim}
Using existing stippling with 4036 points
Image shape: (251, 201)
Generated 42 frames
Point counts: [1, 100, 200, 300, 400, 500, 600, 700, 800, 900, 1000, 1100, 1200, 1300, 1400, 1500, 1600, 1700, 1800, 1900, 2000, 2100, 2200, 2300, 2400, 2500, 2600, 2700, 2800, 2900, 3000, 3100, 3200, 3300, 3400, 3500, 3600, 3700, 3800, 3900, 4000, 4036]
\end{verbatim}

Now let's create the GIF animation:

\subsubsection{Python}

\begin{figure}[H]

{\centering \pandocbounded{\includegraphics[keepaspectratio]{progressive_stippling.gif}}

}

\caption{Progressive stippling animation showing the sequential build-up
of points. Each frame represents an increment of 100 points,
demonstrating how the blue noise stippling pattern develops as more
points are added.}

\end{figure}%




\end{document}
